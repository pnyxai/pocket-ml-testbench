\section{Progress Overview}\label{sec:a}

During the month of April, the socket achieved the following milestones:

\begin{itemize}
    \item Helped in the creation of the \emph{POKT Square RAG Agent}\footnote{\url{https://github.com/pokt-scan/pokt-square}}.
    \item Finished the architectural design of the asynchronic testing procedure (the Pocket \gls{ML} Test-Bench).
    \item Merged issues on the Test-Bench code:
    \begin{itemize}
        \item Template code for Temporal IO on Python~\footnote{\url{https://github.com/pokt-scan/pocket-ml-testbench/issues/9}}.
        \item Created docker-compose files for deploying the test-bench for development~\footnote{\url{https://github.com/pokt-scan/pocket-ml-testbench/issues/12}}.
        \item Combined the task of storing \gls{HF} datasets with the Sampler to create a single Temporal App for both \footnote{\url{https://github.com/pokt-scan/pocket-ml-testbench/issues/18}}.
        \item Added code for saving and retrieving tasks from MongoDB~\footnote{\url{https://github.com/pokt-scan/pocket-ml-testbench/issues/17}}.
        \item Updated the readmes with final software architecture~\footnote{\url{https://github.com/pokt-scan/pocket-ml-testbench/issues/14}}.
        \item Added basic requester code~\footnote{\url{https://github.com/pokt-scan/pocket-ml-testbench/issues/11}}.
    \end{itemize}
    \item A lot of code was also merged code without a matching issue (due to premature stage of the project), it includes:
    \begin{itemize}
        \item Creating a proper logger function for both Go and Python that rungs OK with temporal.
        \item Added initial \emph{Sampler} workflow code in Python.
        \item Cleaned all Readmes for better understanding of the repository.
        \item Make the code more independent from \gls{LMEH}.
        \item Added packages for Pocket RPC and MongoDB connection handling in Go.
    \end{itemize}
\end{itemize}
