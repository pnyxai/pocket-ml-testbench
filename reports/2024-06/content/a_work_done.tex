\section{Progress Overview}\label{sec:a}

During the month of June, the socket achieved the following milestones:

\begin{itemize}[noitemsep]
    \item Merged issues on the Test-Bench code:
    \begin{itemize}[noitemsep]
        \item \textbf{Manager}:
            \begin{itemize}[noitemsep]
                \item Add same trigger restrictions based on blocks. \footnote{\url{https://github.com/pokt-scan/pocket-ml-testbench/pull/67}}
                \item Create logic for tokenizer signature task. \footnote{\url{https://github.com/pokt-scan/pocket-ml-testbench/pull/62}}
            \end{itemize}
        \item \textbf{Sampler}:
            \begin{itemize}[noitemsep]
                \item Create logic for tokenizer signature task. \footnote{\url{https://github.com/pokt-scan/pocket-ml-testbench/pull/57}}
                \item Fix generation\_until in LM classes. \footnote{\url{https://github.com/pokt-scan/pocket-ml-testbench/pull/78}}
            \end{itemize}
        \item \textbf{Evaluator}:
            \begin{itemize}[noitemsep]
                \item Create initial code for lmeh processing. \footnote{\url{https://github.com/pokt-scan/pocket-ml-testbench/pull/60}}
                \item Add signatures tokenizer evaluation. \footnote{\url{https://github.com/pokt-scan/pocket-ml-testbench/pull/63}}
                \item Finish code for lmeh. \footnote{\url{https://github.com/pokt-scan/pocket-ml-testbench/pull/65}}
            \end{itemize}
        \item \textbf{Sidecard}:
            \begin{itemize}[noitemsep]
                \item Added endpoint for tokenizer in llm nodes. \footnote{\url{https://github.com/pokt-scan/pocket-ml-testbench/pull/56}}                
            \end{itemize}
        \item \textbf{General}:
            \begin{itemize}[noitemsep]
                \item Modularizing the tasks to allow signatures. \footnote{\url{https://github.com/pokt-scan/pocket-ml-testbench/pull/54}}
                \item Task configs: update metrics and filters. \footnote{\url{https://github.com/pokt-scan/pocket-ml-testbench/pull/80}}
                \item Added website to show metrics. \footnote{\url{https://github.com/pokt-scan/pocket-ml-testbench/pull/81}}
                \item Full ML Test bench integration. \footnote{\url{https://github.com/pokt-scan/pocket-ml-testbench/pull/83}}            
            \end{itemize}
    \end{itemize}
\end{itemize}

In June 2024, significant progress was made on the socket, culminating in the completion of various tasks and milestones. 
Finally the different components of the \gls{MLTB}, such as the Manager, Sampler, Evaluator, and Sidecard were integrated. 
This integration culminated presenting the leaderboards on a website considering the results of the corresponding task metrics.

In the next sections of the present report the details of these developments and their implications are presented. 
These sections provide an overview of the work completed, updating components initially considered, and the progress made towards the final goal of the socket.

Furthermore, looking ahead, there are several future work ideas to consider. 
These include refining the evaluation metrics further, exploring additional benchmark for extended functionalities. 
These future initiatives aim to enhance the network capacity guided by the state-of-the-art in the field of machine learning and blockchain technology.


\begin{tcolorbox}[colback=red!5!white,colframe=red!75!black]
\textbf{Note:} This report was written \today, and the Huggingface announced an update in the \gls{HFOLML} at Jun 26, 2024. 
In comparison to the socket scope, major change are related to unimplemented tasks, new metrics, and task higly related to chat models. 
On the other hand, there are tasks that just have new configs. 
This report will not cover the new Leaderboard. 
Nevertheles, at POKTscan and PNYX, we are confident that the \gls{MLTB} is ready to be developed in order to reproduce the \textbf{old} Leaderboard. 
furthermore, we except that the \gls{MLTB} will be able in a short time to be updated to the new Leaderboard. 

Addressing evaluations of \gls{LM}, a topic that is at the forefront of knowledge, has an inherently dynamic nature. 
That is to say, the constant (and increasingly accelerated) improvement of the \gls{LM} requires the development of new evaluations that have to be constantly improved. 
What happened with the \gls{HFOLML} is just the first case Pocket Network faces when it wants to support AI in it. 
At POKTscan we alert the community about this inevitable game of cat and mouse, and we encourage everyone to join the game. 
\end{tcolorbox}